\documentclass[xcolor=dvipsnames, size=custom,width=120,height=90,scale=1.17]{beamer}
\usepackage{tikz}

\usetheme{Madrid}
\useinnertheme{circles}


% ====================
% Colors
% ====================

\definecolor{rice_blue}{RGB}{0, 32, 91}
\definecolor{rice_gray}{RGB}{124, 126, 127}

\definecolor{maroon}{RGB}{128, 0, 0}
\definecolor{darkgray}{RGB}{118, 118, 118}
\definecolor{lightgray}{RGB}{214, 214, 206}
\definecolor{yellow}{RGB}{248, 164, 41}
\definecolor{orange}{RGB}{193, 102, 34}
\definecolor{red}{RGB}{143, 57, 49}
\definecolor{lightgreen}{RGB}{145, 171, 90}
\definecolor{darkgreen}{RGB}{88, 89, 63}
\definecolor{blue}{RGB}{21, 95, 131}
\definecolor{violet}{RGB}{53, 14, 32}
\setbeamercolor{palette primary}{bg=rice_blue,fg=white}
\setbeamercolor{palette secondary}{bg=rice_blue,fg=white}
\setbeamercolor{palette tertiary}{bg=rice_blue,fg=white}
\setbeamercolor{palette quaternary}{bg=rice_blue,fg=white}
\setbeamercolor{structure}{fg=rice_blue} % itemize, enumerate, etc
\setbeamercolor{section in toc}{fg=rice_blue} % TOC sections
\setbeamercolor{subsection in head/foot}{bg=rice_gray,fg=white}

% ====================
% Main
% ====================

\title[Abbreviated Title]{Full Title}
\date{\today}


\author[Kevin McCoy]{Kevin McCoy\inst{1,2}}

\institute[Rice University]{\inst{1} Department of Statistics, Rice University \\ \inst{2} The University of Texas, MD Anderson Cancer Center}


\begin{document}
\addtobeamertemplate{headline}{}
{
    \begin{tikzpicture}[remember picture,overlay]
      \node [anchor=north west, inner sep=3mm] at ([xshift=7mm,yshift=0.25mm]current page.north west)
      {\includegraphics[height=6.0mm]{logos/rice_wordmark.eps}};
      
      \node [anchor=north west, inner sep=3mm] at ([xshift=0.0mm,yshift=0.25mm]current page.north west)
      {\includegraphics[height=6.0mm]{logos/rice_shield.eps}};
      
      \node [anchor=north east, inner sep=3mm] at ([xshift=0.75mm,yshift=0.25mm]current page.north east)
      {\includegraphics[height=6.0mm]{logos/mda.png}};
    \end{tikzpicture}
}

%The next statement creates the title page.
\frame{\titlepage}


%---------------------------------------------------------
%This block of code is for the table of contents after
%the title page
\begin{frame}
\frametitle{Table of Contents}
\tableofcontents
\end{frame}
%---------------------------------------------------------



\section{First section}

%---------------------------------------------------------
%Changing visivility of the text
\begin{frame}
\frametitle{Sample frame title}
This is a text in second frame. For the sake of showing an example.

\begin{itemize}
    \item<1-> Text visible on slide 1
    \item<2-> Text visible on slide 2
    \item<3> Text visible on slides 3
    \item<4-> Text visible on slide 4
\end{itemize}
\end{frame}

%---------------------------------------------------------


%---------------------------------------------------------
%Example of the \pause command
\begin{frame}
In this slide \pause

the text will be partially visible \pause

And finally everything will be there
\end{frame}
%---------------------------------------------------------

\section{Second section}

%---------------------------------------------------------
%Highlighting text
\begin{frame}
\frametitle{Sample frame title}

In this slide, some important text will be
\alert{highlighted} because it's important.
Please, don't abuse it.

\begin{block}{Remark}
Sample text
\end{block}

\begin{alertblock}{Important theorem}
Sample text in red box
\end{alertblock}

\begin{examples}
Sample text in green box. The title of the block is ``Examples".
\end{examples}
\end{frame}
%---------------------------------------------------------


%---------------------------------------------------------
%Two columns
\begin{frame}
\frametitle{Two-column slide}

\begin{columns}

\column{0.5\textwidth}
This is a text in first column.
$$E=mc^2$$
\begin{itemize}
\item First item
\item Second item
\end{itemize}

\column{0.5\textwidth}
This text will be in the second column
and on a second tought this is a nice looking
layout in some cases.
\end{columns}
\end{frame}
%---------------------------------------------------------



\end{document}
